\documentclass[10pt,a4paper,BCOR = 12mm,DIV=15]{scrbook}
\usepackage{amsthm}
\usepackage{amsmath}
\usepackage{amssymb}
\usepackage[utf8]{inputenc}
\usepackage{graphicx, subfigure}
\usepackage{placeins} % FloatBarrier
\author{Wolfgang Keller}
\title{The code book}

\newtheorem{Def}{Definition}
\newtheorem{Le}[Def]{Lemma}
\newtheorem{Th}[Def]{Theorem}
\newtheorem{Cor}[Def]{Corollary}
\newtheorem{Rem}[Def]{Remark}
\newtheorem{Ex}[Def]{Example}

\newcommand{\id}{\operatorname{id}}
\newcommand{\inc}{\operatorname{inc}}
\newcommand{\dec}{\operatorname{dec}}
\newcommand{\numgray}{\operatorname{num-to-gray}}
\newcommand{\graynum}{\operatorname{gray-to-num}}

\begin{document}

\chapter{Binary code}

\section{Definitions}

\begin{Def}
Let $k \in \mathbb{Z}_{\geq 2}$ and $n \in \mathbb{Z}_{\geq 0}$. We define $\mathbb{Z}_k^n := \left\{0, \ldots, k-1\right\}^{\left\{0, \ldots, n-1\right\}}$.
\end{Def}

\begin{Def}
Let $a, b \in \mathbb{Z}_k$. We set
\begin{align*}
a \oplus b & := a + b \mod k \\
a \wedge b & := \begin{cases}
1 & a=k-1 \wedge b=k-1 \\
0 & \textnormal{otherwise.}
\end{cases}
\end{align*}
\end{Def}

\section{Incrementing/decrementing integers}

\begin{Def}
We define the \emph{increment} by
\begin{align*}
\inc: \mathbb{Z}_k^n & \rightarrow \mathbb{Z}_k^n: \\
a & \mapsto \left\{a_i \oplus \bigwedge_{j=0}^{i-1} a_j\right\}_{i=0}^{n-1}
\end{align*}
and the \emph{decrement} by
\begin{align*}
\dec: \mathbb{Z}_k^n & \rightarrow \mathbb{Z}_k^n: \\
a & \mapsto \left\{a_i \ominus \bigwedge_{j=0}^{i-1} \overline{a_j}\right\}_{i=0}^{n-1}.
\end{align*}
\end{Def}

\begin{Le}
$\dec \circ \inc = \id$.
\end{Le}
\begin{proof}

\end{proof}

\section{Gray code}

\begin{Def}
We define \emph{num-to-gray} by
\begin{align*}
\numgray: \mathbb{Z}_k^n & \rightarrow \mathbb{Z}_k^n: \\
a & \mapsto \left\{
a_i \ominus a_{i+1} 
\right\}_{i=0}^{n-1}
\end{align*}
where we define $a_n := 0$ and \emph{gray-to-num} by
\begin{align*}
\graynum: \mathbb{Z}_k^n & \rightarrow \mathbb{Z}_k^n: \\
a & \mapsto \left\{
\bigoplus_{j=i}^n a_j
\right\}_{i=0}^{n-1}.
\end{align*}
\end{Def}

\begin{Le}
$\graynum \circ \numgray = \id$.
\end{Le}
\begin{proof}
\end{proof}

\end{document}